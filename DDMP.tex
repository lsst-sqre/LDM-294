\section{DM Discussion and Decision Making Process}
\label{sect:ddmp}

DM has adopted a multi-layered approach to making decisions.
In general, decisions are made at the lowest level possible within the team --- at the level of the individual developer where practical.
When this is not possible, decision making is escalated through the
hierarchy described below.

\subsection{Empowerment}
All DM team members are empowered by the DM Project Manager (PM) and DM Subsystem Scientist (SS) to make decisions on any DM-internal matter, including technical/algorithm issues, process improvements, tool choices, etc., when:
\begin{enumerate}
\item they are willing and able to do the work to implement the decision or with people who agree with the team member,
\item they (collectively) are willing and able to fix any problems if it goes wrong, and
\item they believe that all affected parties (including your immediate manager) would not seriously object to your decision and implementation.
\end{enumerate}

\subsection{RFC Process}
If the above three criteria are not met, perhaps because the team member doesn't know all the affected parties or because they don't know their positions, the team member should publish the proposed decision and implementation as a Jira issue in the Request For Comments (RFC) project with a component of ``DM.''

It is usually difficult to determine all the affected parties for published package interfaces. Changes to interfaces should thus typically go through this process.

It's a good idea to contact any known affected parties before starting this process to check that the resolution is sensible. The institutional technical manager is always affected, as she or he is responsible for tracking the work schedule. If work for others is being proposed, they are obviously affected. The institutional scientist, the DM Software Architect (SA), the DM Interface Scientist (IS), and the DM Subsystem Scientist (SS) are also valuable resources for determining affected parties.

The purpose of an RFC is to inform others about the existence and content of the proposed decision and implementation in order to allow them to evaluate its impact, comment on it, refine it if necessary, and agree (implicitly or explicitly) or object (explicitly) to its execution.

The discussion of the RFC takes place in the medium of the requestor's choosing (e.g., a specific mailing list, the RFC Jira issue itself, a Slack Channel, a convened videocon, some combination of those, etc.), but the requestor should be open to private communications as well.

In the RFC process, the opinions of those who will be doing the work (and fixing any problems if something goes wrong) are given more weight. In some cases, this may mean that the RFC issue's Assignee passes to someone else. The opinions of more senior people or people more experienced in the area should also be given more weight and may also result in the Assignee changing.

The Assignee is responsible for determining when no serious objections remain.  In particular, there is no need to call for a formal vote on the (refined) resolution. If no explicit objections have been raised within, typically, 72 hours for ``ordinary'' issues and 1 week for ``major'' issues, the Assignee should assume that there are none. This is known as ``lazy consensus.'' When this state has been reached, the Assignee is responsible for ensuring that the final consensus has been recorded in the RFC issue before closing it and proceeding with implementation of the decision.

The requestor must be especially careful about not making irreversible changes in the ``lazy consensus'' time period unless they are absolutely certain there's a general agreement on the stated course of action. If something is broken, the requestor must be be ready to fix it. It is critical to apply sound reasoning and good judgment about what may be acceptable and what might be not. Mistakes will happen; accept that occasionally there will be a requirement to revert an action for which it was thought agreement existed.

\subsection{Exceptions and Appeals}
Some proposed resolutions may require changes to one or more of the baselined, change-controlled documents describing the Data Management system (those in DocuShare with an LDM- handle or marked as change-controlled in Confluence).  Note that major changes to budget or scope will almost certainly affect one or more LDM- documents.  In this case only, the DM Configuration Control Board (DMCCB; \secref{sect:dmccb}) may empanel an ad hoc committee including the lead author of the document and other relevant experts. This committee or the CCB itself must \emph{explicitly} approve the change.

Change-controlled documents with other handles, such as LSE- or LPM-, including inter-subsystem interfaces, have project-wide change control processes. Please consult the DM PM, SA, or IS for more information.
At least one member of the DM CCB will read each RFC to determine if it might affect a change-controlled document.

If the DM team can't converge on a resolution to an RFC that has no serious objections but the requestor still feel that something must be done, the request will be escalated. In most non-trivial cases, they will, with the advice of the SA, empanel a group of experts to which they will delegate the right to make the decision, by voting if need be.

\subsection{Formalities}
For project management purposes, RFCs are formally proposals made to the DM PM and SS who by default are responsible for everything in DM (they ``own'' all problems). As owners, they have the final word in accepting or rejecting all proposals. Functionally, they delegate that ownership, the right and responsibility to make decisions -- to others within the team (e.g. the SA, IS, group leads, etc.) who are expected to delegate it even further. Notifying the institutional technical manager about an RFC serves to inform the DM PM.
