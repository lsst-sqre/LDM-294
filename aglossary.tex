% DO NOT EDIT - generated by /Users/womullan/LSSTgit/lsst-texmf/bin/generateAcronyms.py from https://lsst-texmf.lsst.io/.
\newglossaryentry{2D} {name={2D}, description={Two-dimensional}}
\newacronym{AP} {AP} {Alert Production}
\newacronym{API} {API} {Application Programming Interface}
\newglossaryentry{AURA} {name={AURA}, description={\gls{Association of Universities for Research in Astronomy}}}
\newglossaryentry{Alert} {name={Alert}, description={A packet of information for each source detected with signal-to-noise ratio > 5 in a difference image during Prompt Processing, containing measurement and characterization parameters based on the past 12 months of LSST observations plus small cutouts of the single-visit, template, and difference images, distributed via the internet.}}
\newglossaryentry{Alert Production} {name={Alert Production}, description={The principal component of Prompt Processing that processes and calibrates incoming images, performs Difference Image Analysis to identify DIASources and DIAObjects, packages and distributes the resulting Alerts, and runs the Moving Object Processing System.}}
\newglossaryentry{Archive} {name={Archive}, description={The repository for documents required by the NSF to be kept. These include documents related to design and development, construction, integration, test, and operations of the LSST observatory system. The archive is maintained using the enterprise content management system DocuShare, which is accessible through a link on the project website www.project.lsst.org.}}
\newglossaryentry{Archive Center} {name={Archive Center}, description={Part of the LSST Data Management System, the LSST archive center is a data center at NCSA that hosts the LSST Archive, which includes released science data and metadata, observatory and engineering data, and supporting software such as the LSST Software Stack.}}
\newglossaryentry{Association Pipeline} {name={Association Pipeline}, description={An application that matches detected Sources or DIASources or generated Objects to an existing catalog of Objects, producing a (possibly many-to-many) set of associations and a list of unassociated inputs. Association Pipelines are used in Prompt Processing after DIASource generation and in the final stages of Data Release processing to ensure continuity of Object identifiers.}}
\newglossaryentry{Association of Universities for Research in Astronomy} {name={Association of Universities for Research in Astronomy}, description={ consortium of US institutions and international affiliates that operates world-class astronomical observatories, AURA is the legal entity responsible for managing what it calls independent operating Centers, including LSST, under respective cooperative agreements with the National Science Foundation. AURA assumes fiducial responsibility for the funds provided through those cooperative agreements. AURA also is the legal owner of the AURA Observatory properties in Chile.}}
\newglossaryentry{Base Facility} {name={Base Facility}, description={The data center located at the Base Site in La Serena, Chile. The Base Facility is composed of the Base portion of the Prompt Enclave directly supporting Observatory operations, the Commissioning Cluster, an Archive Enclave holding data products, and the Chilean Data Access Center.}}
\newglossaryentry{Batch Production} {name={Batch Production}, description={Computational processing that is executed as inputs become available, in a distributed way across multiple enclaves when needed, while tracking status and outputs. Examples of Batch Production include offline processing for Prompt data products, calibration products, template images, and Special Programs data products. Prioritization protocols for the various types of batch production are given in LDM-148.}}
\newglossaryentry{Butler} {name={Butler}, description={A middleware component for persisting and retrieving image datasets (raw or processed), calibration reference data, and catalogs.}}
\newacronym{CC} {CC} {Change Control}
\newacronym{CCB} {CCB} {\gls{Change Control Board}}
\newglossaryentry{CCD} {name={CCD}, description={\gls{Charge-Coupled Device}}}
\newacronym{CI} {CI} {Continuous Integration}
\newglossaryentry{CMDB} {name={CMDB}, description={Configuration Management Database}}
\newglossaryentry{Calibration Scientist} {name={Calibration Scientist}, description={The person responsible for the system calibration plan who establishes the requirements for the constituent elements of the calibration hardware, software, and operational data. The Calibration Scientist works under the direction of the Systems Engineering group.}}
\newglossaryentry{Camera} {name={Camera}, description={The LSST subsystem responsible for the 3.2-gigapixel LSST camera, which will take more than 800 panoramic images of the sky every night. SLAC leads a consortium of Department of Energy laboratories to design and build the camera sensors, optics, electronics, cryostat, filters and filter exchange mechanism, and camera control system.}}
\newglossaryentry{Center} {name={Center}, description={An entity managed by AURA that is responsible for execution of a federally funded project}}
\newglossaryentry{Change Control} {name={Change Control}, description={The systematic approach to managing all changes to the LSST system, including technical data and policy documentation. The purpose is to ensure that no unnecessary changes are made, all changes are documented, and resources are used efficiently and appropriately.}}
\newglossaryentry{Change Control Board} {name={Change Control Board}, description={ "advisory board to the Project Manager; composed of technical and management representatives who recommend approval or disapproval of proposed changes to}}
\newglossaryentry{Charge-Coupled Device} {name={Charge-Coupled Device}, description={a particular kind of solid-state sensor for detecting optical-band photons. It is composed of a 2-D array of pixels, and one or more read-out amplifiers.}}
\newglossaryentry{Commissioning} {name={Commissioning}, description={A two-year phase at the end of the Construction project during which a technical team a) integrates the various technical components of the three subsystems; b) shows their compliance with ICDs and system-level requirements as detailed in the LSST Observatory System Specifications document (OSS, LSE-30); and c) performs science verification to show compliance with the survey performance specifications as detailed in the LSST Science Requirements Document (SRD, LPM-17).}}
\newglossaryentry{Construction} {name={Construction}, description={The period during which LSST observatory facilities, components, hardware, and software are built, tested, integrated, and commissioned. Construction follows design and development and precedes operations. The LSST construction phase is funded through the \gls{NSF} \gls{MREFC} account.}}
\newacronym{DAC} {DAC} {\gls{Data Access Center}}
\newacronym{DAX} {DAX} {Data Access Services}
\newacronym{DCR} {DCR} {Differential Chromatic Refraction}
\newacronym{DDMPM} {DDMPM} {Data Management Deputy Project Manager}
\newacronym{DIA} {DIA} {Difference Image Analysis}
\newglossaryentry{DIAObject} {name={DIAObject}, description={A DIAObject is the association of DIASources, by coordinate, that have been detected with signal-to-noise ratio greater than 5 in at least one difference image. It is distinguished from a regular Object in that its brightness varies in time, and from a SSObject in that it is stationary (non-moving).}}
\newglossaryentry{DIASource} {name={DIASource}, description={A DIASource is a detection with signal-to-noise ratio greater than 5 in a difference image.}}
\newacronym{DM} {DM} {\gls{Data Management}}
\newglossaryentry{DMCCB} {name={DMCCB}, description={DM Change Control Board}}
\newglossaryentry{DMIS} {name={DMIS}, description={DM Interface Scientist}}
\newglossaryentry{DMLT} {name={DMLT}, description={DM Leadership Team}}
\newacronym{DMPM} {DMPM} {Data Management Project Manager}
\newacronym{DMS} {DMS} {Data Management Subsystem}
\newacronym{DMSR} {DMSR} {DM System Requirements; LSE-61}
\newglossaryentry{DMSS} {name={DMSS}, description={DM Subsystem Scientist}}
\newglossaryentry{DMTN} {name={DMTN}, description={DM Technical Note}}
\newacronym{DOE} {DOE} {\gls{Department of Energy}}
\newacronym{DR} {DR} {Data Release}
\newacronym{DRP} {DRP} {Data Release Production}
\newglossaryentry{Data Access Center} {name={Data Access Center}, description={Part of the LSST Data Management System, the US and Chilean DACs will provide authorized access to the released LSST data products, software such as the Science Platform, and computational resources for data analysis. The US DAC also includes a service for distributing bulk data on daily and annual (Data Release) timescales to partner institutions, collaborations, and LSST Education and Public Outreach (EPO). }}
\newglossaryentry{Data Backbone} {name={Data Backbone}, description={The software that provides for data registration, retrieval, storage, transport, replication, and provenance capabilities that are compatible with the Data Butler. It allows data products to move between Facilities, Enclaves, and DACs by managing caches of files at each endpoint, including persistence to long-term archival storage (e.g. tape).}}
\newglossaryentry{Data Management} {name={Data Management}, description={The LSST Subsystem responsible for the Data Management System (DMS), which will capture, store, catalog, and serve the LSST dataset to the scientific community and public. The DM team is responsible for the DMS architecture, applications, middleware, infrastructure, algorithms, and Observatory Network Design. DM is a distributed team working at LSST and partner institutions, with the DM Subsystem Manager located at LSST headquarters in Tucson.}}
\newglossaryentry{Data Management Subsystem} {name={Data Management Subsystem}, description={The subsystems within Data Management may contain a defined combination of hardware, a software stack, a set of running processes, and the people who manage them: they are a major component of the DM System operations. Examples include the 'Archive Operations Subsystem' and the 'Data Processing Subsystem'"."}}
\newglossaryentry{Data Management System} {name={Data Management System}, description={The computing infrastructure, middleware, and applications that process, store, and enable information extraction from the LSST dataset; the DMS will process peta-scale data volume, convert raw images into a faithful representation of the universe, and archive the results in a useful form. The infrastructure layer consists of the computing, storage, networking hardware, and system software. The middleware layer handles distributed processing, data access, user interface, and system operations services. The applications layer includes the data pipelines and the science data archives' products and services.}}
\newglossaryentry{Data Release} {name={Data Release}, description={The approximately annual reprocessing of all LSST data, and the installation of the resulting data products in the LSST Data Access Centers, which marks the start of the two-year proprietary period.}}
\newglossaryentry{Data Release Production} {name={Data Release Production}, description={An episode of (re)processing all of the accumulated LSST images, during which all output DR data products are generated. These episodes are planned to occur annually during the LSST survey, and the processing will be executed at the Archive Center. This includes Difference Imaging Analysis, generating deep Coadd Images, Source detection and association, creating Object and Solar System Object catalogs, and related metadata.}}
\newglossaryentry{Department of Energy} {name={Department of Energy}, description={cabinet department of the United States federal government; the DOE has assumed technical and financial responsibility for providing the LSST camera. The DOE's responsibilities are executed by a collaboration led by SLAC National Accelerator Laboratory.}}
\newglossaryentry{Difference Image} {name={Difference Image}, description={Refers to the result formed from the pixel-by-pixel difference of two images of the sky, after warping to the same pixel grid, scaling to the same photometric response, matching to the same PSF shape, and applying a correction for Differential Chromatic Refraction. The pixels in a difference thus formed should be zero (apart from noise) except for sources that are new, or have changed in brightness or position. In the LSST context, the difference is generally taken between a visit image and template. }}
\newglossaryentry{Difference Image Analysis} {name={Difference Image Analysis}, description={The detection and characterization of sources in the Difference Image that are above a configurable threshold, done as part of Alert Generation Pipeline.}}
\newglossaryentry{Differential Chromatic Refraction} {name={Differential Chromatic Refraction}, description={The refraction of incident light by Earth's atmosphere causes the apparent position of objects to be shifted, and the size of this shift depends on both the wavelength of the source and its airmass at the time of observation. DCR corrections are done as a part of DIA.}}
\newglossaryentry{Director} {name={Director}, description={The person responsible for the overall conduct of the project; the LSST director is charged with ensuring that both the scientific goals and management constraints on the project are met. S/he is the principal public spokesperson for the project in all matters and represents the project to the scientific community, AURA, the member institutions of LSSTC, and the funding agencies.}}
\newglossaryentry{DocuShare} {name={DocuShare}, description={The trade name for the enterprise management software used by LSST to archive and manage documents}}
\newglossaryentry{Document} {name={Document}, description={Any object (in any application supported by DocuShare or design archives such as PDMWorks or GIT) that supports project management or records milestones and deliverables of the LSST Project}}
\newacronym{EFD} {EFD} {Engineering Facilities Database}
\newglossaryentry{EPO} {name={EPO}, description={Education and Public Outreach}}
\newglossaryentry{Earned Value} {name={Earned Value}, description={A measurement of how much work has been completed compared to how much was expected to have been completed at a given point in the project}}
\newglossaryentry{Enclave} {name={Enclave}, description={Individually defined portions of the computational resources at the Summit, Base, NCSA, and Satellite Facilities, such as the Prompt Enclave, the Archive Enclave, etc. }}
\newacronym{FITS} {FITS} {\gls{Flexible Image Transport System}}
\newglossaryentry{Firefly} {name={Firefly}, description={A framework of software components written by IPAC for building web-based user interfaces to astronomical archives, through which data may be searched and retrieved, and viewed as \gls{FITS} images, catalogs, and/or plots. Firefly tools will be integrated into the Science Platform.}}
\newglossaryentry{Flexible Image Transport System} {name={Flexible Image Transport System}, description={an international standard in astronomy for storing images, tables, and metadata in disk files. See the IAU FITS Standard for details.}}
\newglossaryentry{Handle} {name={Handle}, description={The unique identifier assigned to a document uploaded to DocuShare}}
\newacronym{IAU} {IAU} {International Astronomical Union}
\newglossaryentry{ICBS} {name={ICBS}, description={International Communications and Base Site}}
\newglossaryentry{IPAC} {name={IPAC}, description={No longer an acronym; science and data center at Caltech}}
\newglossaryentry{IRSA} {name={IRSA}, description={Infrared Science Archive}}
\newacronym{IT} {IT} {Integration Test}
\newacronym{ITC} {ITC} {Information Technology Center}
\newglossaryentry{IVOA} {name={IVOA}, description={International Virtual-Observatory Alliance}}
\newacronym{LDF} {LDF} {LSST Data Facility}
\newglossaryentry{LDM} {name={LDM}, description={LSST Data Management (Document Handle)}}
\newglossaryentry{LPM} {name={LPM}, description={LSST Project Management (Document Handle)}}
\newglossaryentry{LSE} {name={LSE}, description={LSST Systems Engineering (Document Handle)}}
\newglossaryentry{LSR} {name={LSR}, description={LSST System Requirements; LSE-29}}
\newacronym{LSST} {LSST} {Large Synoptic Survey Telescope}
\newglossaryentry{LSSTC} {name={LSSTC}, description={\gls{LSST} Corporation. an Arizona 501(c)3 not-for-profit corporation formed in 2003 for the purpose of designing, constructing, and operating the LSST System. During design and development, the Corporation stewarded private funding used for such essential contributions as early site preparation, mirror construction, and early data management system development. During construction, LSSTC will secure private operations funding from international affiliates and play a key role in preparing the scientific community to use the LSST dataset.}}
\newglossaryentry{LaTeX} {name={LaTeX}, description={(Leslie) Lamport TeX (document markup language and document preparation system)}}
\newacronym{MOPS} {MOPS} {Moving Object Processing System}
\newglossaryentry{MREFC} {name={MREFC}, description={\gls{Major Research Equipment and Facility Construction}}}
\newglossaryentry{Major Research Equipment and Facility Construction} {name={Major Research Equipment and Facility Construction}, description={the NSF account through which large facilities construction projects such as LSST are funded}}
\newglossaryentry{Moving Object Processing System} {name={Moving Object Processing System}, description={The Moving Object Processing System (MOPS) identifies new SSObjects using unassociated DIASources. MOPS is part of the Science Pipelines.}}
\newglossaryentry{NASA} {name={NASA}, description={National Aeronautics and Space Administration}}
\newglossaryentry{NCSA} {name={NCSA}, description={National Center for Supercomputing Applications}}
\newacronym{NET} {NET} {Network Engineering Team}
\newacronym{NSF} {NSF} {\gls{National Science Foundation}}
\newglossaryentry{National Science Foundation} {name={National Science Foundation}, description={primary federal agency supporting research in all fields of fundamental science and engineering; NSF selects and funds projects through competitive, merit-based review}}
\newacronym{OCS} {OCS} {Observatory Control System}
\newglossaryentry{OSS} {name={OSS}, description={Observatory System Specifications; LSE-30}}
\newglossaryentry{Object} {name={Object}, description={In LSST nomenclature this refers to an astronomical object, such as a star, galaxy, or other physical entity. E.g., comets, asteroids are also Objects but typically called a Moving Object or a Solar System Object (SSObject). One of the DRP data products is a table of Objects detected by LSST which can be static, or change brightness or position with time.}}
\newglossaryentry{Operations} {name={Operations}, description={The 10-year period following construction and commissioning during which the LSST Observatory conducts its survey}}
\newacronym{PDF} {PDF} {Portable Document Format}
\newacronym{PM} {PM} {Project Manager}
\newacronym{PMCS} {PMCS} {\gls{Project Management Controls System}}
\newacronym{PSF} {PSF} {Point Spread Function}
\newacronym{PST} {PST} {\gls{Project Science Team}}
\newglossaryentry{Project Management Controls System} {name={Project Management Controls System}, description={suite of tools used to organize and manage a project, including cost and schedule databases, a qualified accounting system, and change control}}
\newglossaryentry{Project Manager} {name={Project Manager}, description={The person responsible for exercising leadership and oversight over the entire LSST project; he or she controls schedule, budget, and all contingency funds}}
\newglossaryentry{Project Science Team} {name={Project Science Team}, description={an operational unit within LSST that carries out specific scientific performance investigations as prioritized by the Director, the Project Manager, and the Project Scientist. Its membership includes key scientists on the Project who provide specific necessary expertise. The Project Science Team provides required scientific input on critical technical decisions as the project construction proceeds}}
\newglossaryentry{Project Scientist} {name={Project Scientist}, description={The principal scientific advisor  to the LSST Project Manager to ensure that LSST system specifications are appropriate for achieving the scientific goals of the project; the Project Scientist also works closely with the Systems Engineering group and chairs the LSST Science Council}}
\newglossaryentry{Prompt Processing} {name={Prompt Processing}, description={The processing that occurs at the Archive Center on the nightly stream of raw images coming from the telescope, including Difference Imaging Analysis, Alert Production, and the Moving Object Processing System. This processing generates Prompt Data Products.}}
\newacronym{QA} {QA} {Quality Assurance}
\newacronym{QC} {QC} {Quality Control}
\newglossaryentry{Quality Assurance} {name={Quality Assurance}, description={All activities, deliverables, services, documents, procedures or artifacts which are designed to ensure the quality of DM deliverables. This may include \gls{QC} systems, in so far as they are covered in the charge described in LDM-622. Note that contrasts with the LDM-522 definition of “QA” as “Quality Analysis”, a manual process which occurs only during commissioning and operations. See also: Quality Control.}}
\newglossaryentry{Quality Control} {name={Quality Control}, description={Services and processes which are aimed at measuring and monitoring a system to verify and characterize its performance (as in LDM-522). Quality Control systems run autonomously, only notifying people when an anomaly has been detected. See also Quality Assurance.}}
\newacronym{RA} {RA} {Right Ascension}
\newglossaryentry{REUNA} {name={REUNA}, description={Red Universitaria Nacional}}
\newacronym{RFC} {RFC} {Request For Comment}
\newacronym{RM} {RM} {Release Manager}
\newglossaryentry{Release} {name={Release}, description={With regard to data pipelines or data products, a version that is cleared for distribution (i.e., has met QA specifications), is assigned a version identifier (e.g., 2.1), and does not evolve in the future to enable provenance.}}
\newglossaryentry{Review Hub} {name={Review Hub}, description={An LSST website that acts as a clearinghouse for information about external reviews of all LSST components planned to occur in the next six months. The site links to review-specific websites for both planned reviews and reviews that have been conducted already.}}
\newglossaryentry{Risk} {name={Risk}, description={The degree of exposure to an event that might happen to the detriment of a program, project, or other activity. It is described by a combination of the probability that the risk event will occur and the consequence of the extent of loss from the occurrence, or impact. Risk is an inherent part of all activities, whether the activity is simple and small, or large and complex.}}
\newglossaryentry{Risk Management} {name={Risk Management}, description={The art and science of planning, assessing, and handling future events to avoid unfavorable impacts on project cost, schedule, or performance to the extent possible. Risk management is a structured, formal, and disciplined activity focused on the necessary steps and planning actions to determine and control risks to an acceptable level. Risk Management is an event-based management approach to managing uncertainty.}}
\newacronym{SEM} {SEM} {\gls{Systems Engineering Manager}}
\newglossaryentry{SLAC} {name={SLAC}, description={No longer an acronym; formerly Stanford Linear Accelerator Center}}
\newglossaryentry{SQuaRE} {name={SQuaRE}, description={Science Quality and Reliability Engineering}}
\newglossaryentry{SRD} {name={SRD}, description={LSST Science Requirements; LPM-17}}
\newglossaryentry{SUIT} {name={SUIT}, description={Science User Interface and Tools}}
\newglossaryentry{Science Pipelines} {name={Science Pipelines}, description={The library of software components and the algorithms and processing pipelines assembled from them that are being developed by DM to generate science-ready data products from LSST images. The Pipelines may be executed at scale as part of LSST Prompt or Data Release processing, or pieces of them may be used in a standalone mode or executed through the LSST Science Platform. The Science Pipelines are one component of the LSST Software Stack.}}
\newglossaryentry{Science Platform} {name={Science Platform}, description={A set of integrated web applications and services deployed at the LSST Data Access Centers (DACs) through which the scientific community will access, visualize, and perform next-to-the-data analysis of the LSST data products.}}
\newglossaryentry{Software Stack} {name={Software Stack}, description={Often referred to as the LSST Stack, or just The Stack, it is the collection of software written by the LSST Data Management Team to process, generate, and serve LSST images, transient alerts, and catalogs. The Stack includes the LSST Science Pipelines, as well as packages upon which the DM software depends. It is open source and publicly available.}}
\newglossaryentry{Solar System Object} {name={Solar System Object}, description={A solar system object is an astrophysical object that is identified as part of the Solar System: planets and their satellites, asteroids, comets, etc. This class of object had historically been referred to within the LSST Project as Moving Objects.}}
\newglossaryentry{Source} {name={Source}, description={A single detection of an astrophysical object in an image, the characteristics for which are stored in the Source Catalog of the DRP database. The association of Sources that are non-moving lead to Objects; the association of moving Sources leads to Solar System Objects. (Note that in non-LSST usage "source" is often used for what LSST calls an Object.)}}
\newglossaryentry{Subsystem} {name={Subsystem}, description={A set of elements comprising a system within the larger LSST system that is responsible for a key technical deliverable of the project.}}
\newglossaryentry{Subsystem Manager} {name={Subsystem Manager}, description={responsible manager for an LSST subsystem; he or she exercises authority, within prescribed limits and under scrutiny of the Project Manager, over the relevant subsystem's cost, schedule, and work plans}}
\newglossaryentry{Subsystem Scientist} {name={Subsystem Scientist}, description={The principal science advisor  to a Subsystem Manager; he or she ensures that the subsystem specifications are appropriated for achieving the project's goals}}
\newglossaryentry{Summit} {name={Summit}, description={The site on the Cerro Pachón, Chile mountaintop where the LSST observatory, support facilities, and infrastructure will be built.}}
\newglossaryentry{Summit Facility} {name={Summit Facility}, description={The main Observatory and Auxiliary Telescope buildings at the Summit Site on Cerro Pachon, Chile.}}
\newglossaryentry{Systems Engineer} {name={Systems Engineer}, description={A member of the Systems Engineering group who works closely with the Systems Engineering Manager and the Systems Scientist on the integrated LSST system's various technical issues spanning the full life cycle of the entire project}}
\newglossaryentry{Systems Engineering} {name={Systems Engineering}, description={an interdisciplinary field of engineering that focuses on how to design and manage complex engineering systems over their life cycles. Issues such as requirements engineering, reliability, logistics, coordination of different teams, testing and evaluation, maintainability and many other disciplines necessary for successful system development, design, implementation, and ultimate decommission become more difficult when dealing with large or complex projects. Systems engineering deals with work-processes, optimization methods, and risk management tools in such projects. It overlaps technical and human-centered disciplines such as industrial engineering, control engineering, software engineering, organizational studies, and project management. Systems engineering ensures that all likely aspects of a project or system are considered, and integrated into a whole.}}
\newglossaryentry{Systems Engineering Manager} {name={Systems Engineering Manager}, description={individual responsible for the oversight and coordination of the LSST systems engineering efforts as well as the management of the Systems Engineering group and work package. The SEM is also the CCB Chair and as such is responsible for the execution, technical oversight, and coordination of configuration control activities.}}
\newglossaryentry{Systems Scientist} {name={Systems Scientist}, description={A member of the Systems Engineering group and chief liaison to all project scientists; the Systems Scientist works closely with the Systems Engineering Manager and is responsible for the flow-down of science requirements. The Systems Scientist ensures that acceptance testing and commissioning address the science requirements.}}
\newglossaryentry{T/CAM} {name={T/CAM}, description={Technical/Control (or Cost) Account Manager}}
\newacronym{US} {US} {United States}
\newglossaryentry{Validation} {name={Validation}, description={A process of confirming that the delivered system will provide its desired functionality; overall, a validation process includes the evaluation, integration, and test activities carried out at the system level to ensure that the final developed system satisfies the intent and performance of that system in operations}}
\newglossaryentry{Verification} {name={Verification}, description={The process of evaluating the design, including hardware and software - to ensure the requirements have been met;  verification (of requirements) is performed by test, analysis, inspection, and/or demonstration}}
\newacronym{WBS} {WBS} {\gls{Work Breakdown Structure}}
\newacronym{WCS} {WCS} {\gls{World Coordinate System}}
\newacronym{WG} {WG} {Working Group}
\newglossaryentry{WISE} {name={WISE}, description={Wide-field Survey Explorer}}
\newglossaryentry{Work Breakdown Structure} {name={Work Breakdown Structure}, description={a tool that defines and organizes the LSST project's total work scope through the enumeration and grouping of the project's discrete work elements}}
\newglossaryentry{World Coordinate System} {name={World Coordinate System}, description={a mapping from image pixel coordinates to physical coordinates; in the case of images the mapping is to sky coordinates, generally in an equatorial (RA, Dec) system. The \gls{WCS} is expressed in FITS file extensions as a collection of header keyword=value pairs (basically, the values of parameters for a selected functional representation of the mapping) that are specified in the FITS Standard.}}
\newglossaryentry{afw} {name={afw}, description={LSST's pipeline library code and primitives including images and tables.}}
\newglossaryentry{aggregate metric} {name={aggregate metric}, description={An aggregation of multiple point metrics. For example, the overall photometric repeatability for a particular tract given given the repeatability of multiple individual stars in the tract. See also: “metric”.}}
\newglossaryentry{aggregation} {name={aggregation}, description={The process of reducing multiple input values to a single output, e.g., a metric value, computed from a collection of input values. For example, a sum or average of a metric computed over patches to produce an aggregate metric at tract level. See also: “metric”, “aggregate metric”.}}
\newglossaryentry{airmass} {name={airmass}, description={The pathlength of light from an astrophysical source through the Earth's atmosphere. It is given approximately by sec z, where z is the angular distance from the zenith (the point directly overhead, where airmass = 1.0) to the source.}}
\newglossaryentry{astronomical object} {name={astronomical object}, description={A star, galaxy, asteroid, or other physical object of astronomical interest. Beware: in non-LSST usage, these are often known as sources.}}
\newglossaryentry{background} {name={background}, description={In an image, the background consists of contributions from the sky (e.g., clouds or scattered moonlight), and from the telescope and camera optics, which must be distinguished from the astrophysical background. The sky and instrumental backgrounds are characterized and removed by the LSST processing software using a low-order spatial function whose coefficients are recorded in the image metadata.}}
\newglossaryentry{calibration} {name={calibration}, description={The process of translating signals produced by a measuring instrument such as a telescope and camera into physical units such as flux, which are used for scientific analysis. Calibration removes most of the contributions to the signal from environmental and instrumental factors, such that only the astronomical component remains.}}
\newglossaryentry{camera} {name={camera}, description={An imaging device mounted at a telescope focal plane, composed of optics, a shutter, a set of filters, and one or more sensors arranged in a focal plane array.}}
\newglossaryentry{configuration} {name={configuration}, description={A task-specific set of configuration parameters, also called a 'config'. The config is read-only; once a task is constructed, the same configuration will be used to process all data. This makes the data processing more predictable: it does not depend on the order in which items of data are processed. This is distinct from arguments or options, which are allowed to vary from one task invocation to the next.}}
\newglossaryentry{flux} {name={flux}, description={Shorthand for radiative flux, it is a measure of the transport of radiant energy per unit area per unit time. In astronomy this is usually expressed in cgs units: erg/cm2/s.}}
\newglossaryentry{git} {name={git}, description={A distributed revision control system, often used for software source code. See the Git User Manual for details. Not developed by LSST DM.}}
\newglossaryentry{metadata} {name={metadata}, description={General term for data about data, e.g., attributes of astronomical objects (e.g. images, sources, astroObjects, etc.) that are characteristics of the objects themselves, and facilitate the organization, preservation, and query of data sets. (E.g., a FITS header contains metadata).}}
\newglossaryentry{metric} {name={metric}, description={A measurable quantity which may be tracked. A metric has a name, description, unit, references, and tags (which are used for grouping). A metric is a scalar by definition. See also: aggregate metric, model metric, point metric.}}
\newglossaryentry{metric value} {name={metric value}, description={The result of computing a particular metric on some given data. Note that metric values are typically computed rather than measured. See also: metric.}}
\newglossaryentry{model metric} {name={model metric}, description={A metric describing a model related to the data. For example, the coefficients of a 2D polynomial fit to the background of a single CCD exposure.}}
\newglossaryentry{monitoring} {name={monitoring}, description={In DM QA, this refers to the process of collecting, storing, aggregating and visualizing metrics.}}
\newglossaryentry{patch} {name={patch}, description={An quadrilateral sub-region of a sky tract, with a size in pixels chosen to fit easily into memory on desktop computers.}}
\newglossaryentry{pipeline} {name={pipeline}, description={A configured sequence of software tasks (Stages) to process data and generate data products. Example: Association Pipeline.}}
\newglossaryentry{point metric} {name={point metric}, description={A metric that is associated with a single entry in a catalog. Examples include the shape of a source, the standard deviation of the flux of an object detected on a Coadd, the flux of an source detected on a difference image.}}
\newglossaryentry{provenance} {name={provenance}, description={Information about how LSST images, Sources, and Objects were created (e.g., versions of pipelines, algorithmic components, or templates) and how to recreate them.}}
\newglossaryentry{shape} {name={shape}, description={In reference to a Source or Object, the shape is a functional characterization of its spatial intensity distribution, and the integral of the shape is the flux. Shape characterizations are a data product in the DIASource, DIAObject, Source, and Object catalogs.}}
\newglossaryentry{sky map} {name={sky map}, description={A sky tessellation for LSST. The Stack includes software to define a geometric mapping from the representation of World Coordinates in input images to the LSST sky map. This tessellation is comprised of individual tracts which are, in turn, comprised of patches.}}
\newglossaryentry{stack} {name={stack}, description={A record of all versions of a document uploaded to a particular DocuShare handle}}
\newglossaryentry{tract} {name={tract}, description={A portion of sky, a spherical convex polygon, within the LSST all-sky tessellation (sky map). Each tract is subdivided into sky patches.}}
\newglossaryentry{transient} {name={transient}, description={A transient source is one that has been detected on a difference image, but has not been associated with either an astronomical object or a solar system body.}}
