\section{Roles in \gls{Data Management}}

This section describes the responsibilities associated with the roles shown in
\figref{fig:dmorg}.


\subsection{DM \gls{Project Manager} (\gls{DMPM})\label{role:dmpm}}

The \gls{DM} \gls{Project Manager} is responsible for the efficient coordination of all \gls{LSST} activities and responsibilities assigned to the \gls{Data Management} \gls{Subsystem}. The \gls{DM} \gls{Project Manager} has the responsibility of establishing the organization, resources, and work assignments to provide \gls{DM} solutions.  The \gls{DM} \gls{Project Manager} serves as the \gls{DM} representative in the \gls{LSST} Project Office and in that role is responsible for presenting \gls{DM} initiative status and submitting new \gls{DM} initiatives to be considered for approval. Ultimately, the \gls{DM} \gls{Project Manager}, in conjunction with his/her peer Project Managers (Telescope, \gls{Camera}), is responsible for delivering an integrated \gls{LSST} system. The \gls{DM} \gls{Project Manager} reports to the \gls{LSST} \gls{Project Manager}. Specific responsibilities include:

\begin{itemize}
\item Manage the overall \gls{DM} System
\item Define scope and request funding for \gls{DM} System
\item Develop and implement the \gls{DM} project management and control process, including earned value management
\item Approve the \gls{DM} \gls{Work Breakdown Structure} (\gls{WBS}), budgets and resource estimates
\item Approve or execute as appropriate all \gls{DM} outsourcing contracts
\item Convene and/or participate in all \gls{DM} reviews
\item Co-chair the \gls{DM} Leadership Team (\secref{sect:dmlt})
\end{itemize}

\subsection{Deputy DM \gls{Project Manager} (\gls{DDMPM}) \label{role:ddmpm}}
The PM and deputy will work together on the general management of DM and any specific PM tasks may be delegated to the deputy as needed and agreed. In the absence of the PM the deputy carries full authority and decision making powers of the PM. The DM Project Manager will keep the Deputy Project Manager informed of all DM situations such that the deputy may effectively act in place of the Project Manager when absent.

\subsection{DM \gls{Subsystem Scientist} (\gls{DMSS}) \label{role:dmps} }

The DM \gls{Subsystem Scientist} (\gls{DMSS}) has the ultimate responsibility for ensuring DM initiatives provide solutions that meet the overall \gls{LSST} science goals. As such, this person leads the definition and understanding of the science goals and deliverables of the \gls{LSST} \gls{Data Management} System and is accountable for communicating these to the DM engineering team.

The DM \gls{Subsystem Scientist} reports to the \gls{LSST} \gls{Project Scientist}. The \gls{DMSS} is a member of the \gls{LSST} \gls{Change Control Board} and the \gls{Project Science Team}. He/she chairs and directs the work of the DM System Science Team (\secref{sect:dmsst}).

Specific responsibilities and authorities include:


\begin{itemize}
\item Communicates with \gls{DM} science stakeholders (\gls{LSST} \gls{Project Scientist} and Team, advisory bodies, the science community) to understand their needs and identifies aspects to be satisfied by the \gls{DM} \gls{Subsystem}.
\item Develops, maintains, and articulates the vision of \gls{DM} products and services responsive to stakeholder needs.
\item Works with the \gls{LSST} \gls{Project Scientist} to communicate the \gls{DM} System vision to \gls{DM} stakeholders. Works with the \gls{DM} \gls{Project Manager} to communicate and articulate the \gls{DM} System vision and requirements to the \gls{DM} construction team.
\item Regularly monitors \gls{DM} construction team progress and provides feedback to the \gls{DM} \gls{Project Manager} to ensure the continual understanding of and adherence to the \gls{DM} vision, requirements, and priorities.
\item Develops and/or evaluates proposed changes to \gls{DM} deliverables driven by schedule, budget, or other constraints.
\item Provides advice to the \gls{DM} \gls{Project Manager} on science-driven prioritization of construction activities.
\item Validates the science quality of \gls{DM} deliverables and the capability of all elements of the \gls{DM} System to achieve \gls{LSST} science goals.
\item Serves as \gls{Data Management} Liaison as requested by \gls{LSST} Science Collaborations
\item Provides safe, effective, efficient operations in a respectful work environment.
\end{itemize}

Specific authorities include:

\begin{itemize}
\item Defines the vision and high-level requirements of the \gls{DM} products and services required to deliver on \gls{LSST} science goals.
\item Defines the science acceptance criteria for \gls{DM} deliverables (both final and intermediate) and validates that they have been met (Science \gls{Validation}).
\item Hires or appoints \gls{DM} System Science Team staff and other direct reports and defines their responsibilities.
\item Advises and consents to the appointments of institutional \gls{DM} Science Leads.
\item Delegates authority and responsibility as appropriate to institutional Science Leads and other members of the DM System Science Team.
\item Represents and speaks for the \gls{LSST} \gls{Data Management}.
\item Convenes and/or participates in all \gls{DM} reviews.
\item Co-Chairs the \gls{DM} Leadership Team
\end{itemize}

\subsection{Deputy DM \gls{Subsystem Scientist} (\gls{DMSS}) \label{role:ddmss} }

The relationship the \gls{Subsystem Scientist} and deputy is equivalent to that
between the \gls{Project Manager} and deputy (see \secref{role:ddmpm}).


\subsection{Project Controller/Scheduler \label{role:pcon}}

The \gls{DM} Project Controller is responsible for integrating \gls{DM}'s agile planning process with the \gls{LSST} Project Management and Control System (\gls{PMCS}). Specific responsibilities include:

\begin{itemize}

  \item{Assist T/CAMs in developing the \gls{DM} plan}
  \item{Synchronize the \gls{DM} plan, managed as per \secref{sect:plan}, with the \gls{LSST} PMCS}
  \item{Ensure that the plan is kept up-to-date and milestones are properly tracked}
  \item{Create reports, Gantt charts and figures as requested by the \gls{DMPM}}

\end{itemize}

\subsection{Product Owner \label{role:prodo}}

A product owner is responsible for the quality and acceptance of a particular product.
The product owner shall sign off on the requirements to be fulfilled in every delivery and therefore also on any descopes or enhancements.
The product owner shall define tests which can be run to prove a delivery meets the requirements due for that product.

\subsection{Senior advisor / Pipelines Scientist \label{role:pipe}}

The Senior advisor has direct communicaitons with the \gls{PM}. Several \gls{DM} products come together to form the \gls{LSST} \gls{pipeline}. The Pipelines Scientist is the product owner for the overall \gls{pipeline}.

The Pipelines Scientist shall:

\begin{itemize}

\item Provide guidance and test criteria for the full \gls{pipeline} including how \gls{QA} is done on the products
\item Keep the big picture of where the codes are going in view, predominantly with respect to the algorithms, but also the implementation and architecture (as part of the \gls{Systems Engineering} Team \secref{sect:sysengt}).
\item Advise on how we should attack algorithmic problems, providing continuing advice to subsystem product owners as we try new things.
\item Advise on \gls{calibration} issues, provide understanding of the detectors from a \gls{DM} point of view
\item Advise on the overall (scientific) performance of the system, and how we'll test it, thinking about all the small things that we have to get right to make the overall system good.

\end{itemize}

\subsection{Science Platform Scientist \label{role:scip}}
The science platform is composed of three aspects. Each aspect is produced in a different institution.
Each aspect has its own science lead/product owner.
The product owner for the platform is the \gls{DM} \gls{Subsystem Scientist} \secref{role:dmps} with final say on requirements and features, however since this is a vital tool for \gls{LSST} science we feel it is also important to have a scientist considering the platform as a whole.
Hence this role is to be the scientific guardian of the science platform as a whole, to make sure all of the aspects work together in a useful manner allowing scientific exploitation of the \gls{LSST} data. The \gls{Science Platform} Scientist works in close collaboration with the \gls{DM} \gls{Subsystem Scientist}.

\subsection{Systems Engineer \label{role:sysengineer}}

With the \gls{Systems Engineering} Team (\secref{sect:sysengt}) the \gls{Systems Engineer} owns the \gls{DM} entries in the risk register and is generally in charge of the \textit{process} of building \gls{DM} products.

As such, the \gls{Systems Engineer} is responsible for managing requirements as they pertain to \gls{DM}.
This includes:

\begin{itemize}
\item Update and ensure traceability of the high level design \& requirements documents: \gls{DMSR} (\citeds{LSE-61}), \gls{OSS} (\citeds{LSE-30}), and \gls{LSR} (\citeds{LSE-29})
\item Oversee work on lower level requirements documents
\item Ensure  that the system is appropriately modeled in terms of e.g. drawings, design documentation, etc
\item Ensure  that solid verification plans and standards are established within \gls{DM}
\end{itemize}

In addition, the \gls{Systems Engineer} is responsible for the process to define \& maintain \gls{DM} interfaces (internal and external)

\begin{itemize}
\item Define and enforce standards for internal interfaces
\item Direct the Interface Scientist's (\secref{role:dmis}) work on external ICDs
\end{itemize}

The \gls{Systems Engineer} shall chair the \gls{DM} \gls{Change Control Board} (\secref{sect:dmccb})

\begin{itemize}
\item Organize \gls{DMCCB} processes so that the change control process runs smoothly
\item Identify RFCs requiring \gls{DMCCB} attention
\item Shepherd RFCs through change control
\item Call and chair \gls{DMCCB} meetings, ensuring that decisions are made and recorded
\end{itemize}

Finally, the \gls{Systems Engineer} represents \gls{DM} on the \gls{LSST} \gls{CCB}.

\subsection{DM Interface Scientist (\gls{DMIS}) \label{role:dmis}}

The \gls{DM} Interface Scientist is responsible for all external interfaces to the \gls{DM} \gls{Subsystem}. This includes ensuring that appropriate tests for those interfaces are defined. This is a responsibility delegated from the \gls{DM} \gls{Systems Engineer} (\secref{role:sysengineer}).

As we begin to implement these interfaces this role will diminish as implementers take up the ownership of the interfaces.

\subsection{Software Architect \label{role:softarc}}

The Software Architect is responsible for the overall design of the \gls{DM} \textit{software} system. Specific responsibilities include:

\begin{itemize}

\item{Define the overall architecture of the system and ensuring that all products integrate to form a coherent whole}
\item{Select and advocate appropriate software engineering techniques}
\item{Choose the technologies which are used within the codebase}
\item{Minimize the exposure of \gls{DM} to volatile external dependencies}

\end{itemize}

The Software Architect will work closely with the \gls{Systems Engineer} (\secref{role:sysengineer}) to ensure that processes are in place for tracing requirements to the codebase and providing hooks to ensure that requirement verification is possible.

\subsection{Operations Architect \label{role:opsarc}}

The \gls{DM} \gls{Operations} Architect is responsible for ensuring that all elements of the \gls{DM} \gls{Subsystem}, including operations teams, infrastructure, middleware, applications, and interfaces,
come together to form an operable system.

Specific responsibilities include:

\begin{itemize}
\item Set up and coordinate operations rehearsals
\item Ensure readiness of procedures and personnel for \gls{Operations}
\item Set standards for operations e.g. procedure handling and operator logging
\item Participate in stakeholder and end user coordination and approval processes and reviews
\item Serve as a member of the \gls{LSST} \gls{Systems Engineering} Team
\end{itemize}

\subsection{Release Manager (\gls{RM})}\label{role:dmrm}

The \gls{DM} \gls{Release} Manager (\gls{RM}) is responsible for maintaining and applying the release policy.
Specifically, the \gls{DM} \gls{Release} Manager will:

\begin{itemize}

  \item{Develop and maintain the \gls{DM} \gls{Release} Policy as a change controlled
  document;}
  \item{Manage the software release process and its compliance with documented
  policy;}
  \item{Define the contents of releases, in conjunction with the product
  owners, the \gls{DM} \gls{Subsystem Scientist}, and the technical managers;}
  \item{Ensure that each release is accompanied by an appropriate
  documentation pack, including user manuals, test specifications and reports,
  and release notes;}
  \item{Ensure the release is delivered to \gls{NCSA} for acceptance;}
  \item{Work with technical managers to coordinate bug fixes and maintenance
  of long-term support releases;}
  \item{Serve as a member of the \gls{DMCCB} (\secref{sect:dmccb}).}

\end{itemize}

\subsection{Lead Institution Senior Positions}

Each Lead Institution (as defined in \secref{sect:leadtutes}; see also \tabref{tab:wbs}) has a \gls{T/CAM} and Scientific or Engineering Lead, who jointly have overall responsibility for a broad area of \gls{DM} work, typically a \gls{Work Breakdown Structure} (\gls{WBS}) Level 2 element. They are supervisors of the team at their institution, with roles broadly analogous to those of the \gls{DM} \gls{Project Manager} and \gls{Subsystem Scientist}.

\subsubsection{Technical/Control Account Manager (\gls{T/CAM}) \label{role:tcam}}

Technical/Control Account Managers have managerial and financial responsibility
for the engineering teams within \gls{DM}. Each \gls{T/CAM} is responsible for a specific set of \gls{WBS} elements. Their detailed responsibilities include:

\begin{itemize}

  \item{Develop, resource load, and maintain the plan for executing the \gls{DM} construction project within the scope of their WBS}
  \item{Synchronize the construction schedule with development in \gls{WBS} elements managed by other T/CAMs}
  \item{Maintain the budget for their \gls{WBS} and ensuring that all work undertaken is charged to the correct accounts}
  \item{Work with the relevant Science Leads and Product Owners (\secref{role:prodo}) to develop the detailed plan for each cycle and sprint as required}
  \item{Work with the \gls{DM} Project Controller (\secref{role:pcon}) to ensure that all plans and milestones are captured in the \gls{LSST} Project Controls system}
  \item{Perform day-to-day management of staff within their \gls{WBS}}
  \item{Perform the role of ``scrum-master'' during agile development}
  \item{Report activities as required, including providing input for monthly status reports.}

\end{itemize}

\subsubsection{Institutional Science/Engineering Lead \label{role:scilead}}

The Institutional Science/Engineering Leads serve as product owners (\secref{role:prodo}) for the major components of the \gls{DM} System (\gls{Alert Production}, Data \gls{Release} Production, Science User Interface etc).

In addition, they provide scientific and technical expertise to their local engineering teams.

They work with the \gls{T/CAM} who has managerial responsibility for their product to define the overall construction plan and the detailed cycle plans for \gls{DM}.

Institutional science leads are members of the \gls{DM} System Science Team (\secref{sect:dmsst}) and, as such, report to the \gls{DM} \gls{Subsystem Scientist} (\secref{role:dmps}).

\subsection{DM Science \gls{Validation} Scientist}
\label{role:dmsvs}

The \gls{DM} Science \gls{Validation} Scientist leads the Science \gls{Validation} team (\secref{sect:dmsvt}).
This individual has primary responsibility for planning, executing and analyzing the results of science validation activities, as defined in \citeds{LDM-503}; typically, this includes large-scale data challenges.
The Science \gls{Validation} Scientist is responsible for End to End Science validation and reports to the \gls{DM} \gls{Subsystem Scientist}.

\subsection{Cross-Cutting Roles}\label{role:crosscut}

There are at least two roles which involve managing work across institute and \gls{WBS} boundaries.
These individuals act as coordinators for the cross-cutting activity, including organizing ``standup'' (or other) meetings and resolving technical difficulties.
They should develop a master schedule for activities within their area of responsibility and synchronize it with the T/CAMs who are managing individual teams.
Day-to-day management of staff resides with the \gls{T/CAM} of the appropriate \gls{WBS}; it follows that stories can only be assigned to individuals with the agreement of that \gls{T/CAM}.
Though this is more of a coordination-oriented role, these managers have authority to prioritize stories in the relevant area.

\subsubsection{Science Platform Manager}\label{role:lsplead}

The \gls{LSST} \gls{Science Platform} spans multiple \gls{WBS} elements bringing together authentication, front-end services, database access, and notebook execution.
At time of writing, Frossie Economou is the \gls{Science Platform} Manager.

\subsubsection{Middleware Manager}\label{role:mwlead}

Middleware covers several \gls{WBS} elements and requires multiple parts of the system to work in unison.
This includes task execution, workflow management, data access abstractions (the ``Data \gls{Butler}''), and \gls{provenance}.
At time of writing, Fritz Mueller is the Middleware Manager.
