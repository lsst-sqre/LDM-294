\section{Data Management Conceptual Architecture \label{sect:dmarc}}

The \gls{DM} \gls{Subsystem} Architecture is detailed in \citeds{LDM-148}.
A few of the higher level diagrams are reproduced here to orient the reader within \gls{DM}.

During \gls{Operations}, components of the \gls{DM} \gls{Subsystem} will be installed and run in
multiple locations. These include:

\begin{itemize}
\item The \gls{Commissioning} Cluster in the \gls{Base Facility} in La Serena, Chile
\item The commissioning/construction compute facility at \gls{NCSA} in Urbana-Champaign
\item The \gls{USDF} at \gls{SLAC} in Menlo Park
\item The \gls{IDF} running on Google
\item The \gls{US} \gls{DAC},  at \gls{SLAC} in Menlo Park
\item The Chilean \gls{DAC} in the \gls{Base Facility}
\item The \gls{FrDF} at \gls{CC-IN2P3} in Lyon, France
\end{itemize}

\figref{fig:dmsdeploy} shows the various \gls{DM} components which will be used in operations and the physical compute environments in which they will be deployed.
Bulk data storage and transport between components is provided by the \gls{Data Backbone}. This complex piece of infrastructure is displayed in \figref{fig:databb}.

Science users will access the data products produced by \VRO~ through the
Science Platform, as shown in \figref{fig:sciplat}.

\figref{fig:dcs} shows the common infrastructure and services layer which underlies the compute environments.
This does not list specific technologies for management/monitoring, provisioning/deployment, or workload/workflow --- these are still being selected --- but under consideration are industry-standard tools such as Nagios, Puppet/vSphere/OpenStack/Kubernetes, and Pegasus.

\begin{figure}[htbp]
\begin{center}
\includegraphics[width=1.0\textwidth, trim={0cm 4.5cm 0cm 0cm}]{images/DMSDeployment}
\caption{DM components as deployed during \gls{Operations}. Refer to \citeds{LDM-148} for details of each component.}
\label{fig:dmsdeploy}
\end{center}
\end{figure}

\begin{figure}[htbp]
\begin{center}
%\includegraphics[width=0.5\textwidth]{images/SciencePlatform}
\includegraphics[width=0.9\textwidth]{images/fig-rubin-science-platform}
\caption{The sub-components of the \gls{Science Platform}. \label{fig:sciplat}}
\end{center}
\end{figure}


\begin{figure}[htbp]
\begin{center}
\includegraphics[width=0.6\textwidth]{images/DataBackbone}
\caption{The \gls{Data Backbone} links all the physical components of \gls{DM}. \label{fig:databb}}
\end{center}
\end{figure}

\begin{figure}[htbp]
\begin{center}
 \includegraphics[width=\textwidth, trim={0cm 14cm 0cm 0cm}]{images/DMSCommonServices}
\caption{Common infrastructure services available at \gls{DM} locations. \label{fig:dcs}}
\end{center}
\end{figure}



\subsection{External Interfaces \& Auxiliary Data}
The \gls{DM} external interfaces are controlled by the ICDs listed in \tabref{tab:icds}.

\begin{table}
    \begin{center}
      \caption{DM Interface Control Documents \label{tab:icds}}
      \begin{tabular}{l p{0.7\textwidth}}
          \hline
          \citeds{LSE-68} & Data Acquisition Interface between \gls{Data Management} and \gls{Camera}\\
          \citeds{LSE-69} & Interface between the \gls{Camera} and \gls{Data Management}   \\
          \citeds{LSE-72} & \gls{OCS} Command Dictionary for \gls{Data Management}\\
          \citeds{LSE-75} & Control System Interfaces between the Telescope and \gls{Data Management}\\
          \citeds{LSE-76} & Infrastructure Interfaces between \gls{Summit Facility} and \gls{Data Management}\\
          \citeds{LSE-77} & Infrastructure Interfaces between \gls{Base Facility} and \gls{Data Management}\\
          \citeds{LSE-130} & List of Data Items to be Exchanged Between the \gls{Camera} and \gls{Data Management}\\
          \citeds{LSE-131} & \gls{Data Management} Interface Requirements to Support Education and Public Outreach \\
          \citeds{LSE-140} & Auxiliary Instrumentation Interface between \gls{Data Management} and Telescope\\
          \hline
      \end{tabular}
    \end{center}
\end{table}

In addition, certain tasks in \gls{DM} rely on external catalogs and other information.
The current design requires:
\begin{itemize}
\item Gaia catalog (\gls{Release} 2) as a photometry baseline.
\end{itemize}
