\section{Teams within Data Management} \label{sect:groups}

Since the DM team is distributed in terms of geography and responsibility across the LSST partner and lead institutions, mechanisms are needed to ensure that the project remains on track at all times. There are five primary coordinating bodies to ensure the management, technical, and quality integrity of the DM Subsystem.

\subsection{System Science Team \label{sect:dmsst}}

Members of the DM System Science Team (SST) work together to define, maintain, and communicate to the DM Systems Engineering team a coherent vision of the LSST DM system responsive to the overall LSST Project goals, as well as scientifically validate the as-built system (\citeds{LDM-503}, Section~9.).

\begin{figure}[htbp]
\begin{center}
\includegraphics[width=0.8\textwidth]{images/DmSSTOrg}
\caption{DM System Science Team organisation.
\label{fig:sstorg}}
\end{center}
\end{figure}



\subsubsection{Organization and Goals}
\label{sect:dm-sst-org}

The System Science Team includes:
\begin{itemize}
\item \gls{DMSS} (chair)
\item DM Science Validation Scientist
\item DM Institutional Science Leads
\item DM System Science Analysts
\item DM Science Pipelines Scientist
\end{itemize}

The System Science Team has been chartered to:
\begin{itemize}
\item Support the \gls{DMSS} (as the overall DM Product Owner) in ensuring that Data Management Subsystem's initiatives provide solutions that meet the overall LSST science goals.
\item Support the Institutional Science Leads in their roles as Product Owners for elements of the DM system their respective institutions have been tasked to deliver.
\item Support the DM Science Validation Scientist, who organizes and coordinates the science validation efforts (\citeds{LDM-503}).
\item Guide the work of System Science Analysts, who generally lead and/or execute studies needed to support SST work.
\item Provide a venue for communication with the Science Pipelines Scientist, who broadly advises on topics related to the impact of science pipelines on delivered science and vice versa (\secref{role:pipe}).
\end{itemize}

The members of the System Science Team report to the \gls{DMSS} and share the following responsibilities:
\begin{itemize}
\item Communicate with the science community and internal stakeholders to understand their needs, identifying the aspects to be satisfied by the DM Subsystem.
\item Liaise with the science collaborations to understand and coordinate any concurrent science investigations relevant to the DM Subsystem.
\item Develop, maintain, and articulate the vision of DM-delivered LSST data products and services that is responsive to stakeholder needs, balanced across science areas, well motivated, and scientifically and technologically current.
\item Work with the \gls{DMPM} and DM \glspl{T/CAM} to communicate and articulate the DM System vision and requirements to the DM engineering team.
\item Identify, develop, and champion new scientific opportunities for the LSST DM System, as well as identify risks where possible.
\item Develop change proposals and/or evaluate the scientific impact of proposed changes to DM deliverables driven by schedule, budget, or other constraints.
\item Lead the Science Verification of the deliverables of the DM subsystem.
\end{itemize}

\subsubsection{Regression Monitoring of KPMs and other Metrics}

All KPMs and other regression monitoring metrics will be calculated on a regular cadence (daily if possible).
They are monitored by the SQuaRE scientist, with status periodically reported to the System Science Team (SST).
The SQuaRE scientist brings up any major regressions to the attention of the SST, along with an initial assessment of the problem.
The SST has the responsibility of monitoring the overall system for whether it meets its key performance metrics as well as understanding any significant performance regressions in performance.
The SST may recommend further actions to the \gls{DMPM} and/or \gls{DMSS}, if necessary.
These include performing additional testing, broader root cause analysis, documenting the regression, or recommendations on the priority of fixing the regression relative to presently scheduled work.

\subsubsection{Communications}

DM System Science Team communication mechanisms are described on the SST Confluence page at \url{http://ls.st/sst}. The list of current  DM liaisons to the LSST Science Collaborations and international partners is maintained  in  \url{https://www.lsstcorporation.org/science-collaborations}

\subsubsection{Time Allocation for Institutional Science Leads}

The Institutional Science Leads fulfill the role of \textit{Product Owner} for elements of the DM system that their respective institutions have been tasked to deliver; institutional T/CAMs rely on their Scientist to provide  \emph{Product Owner}  services.
In addition, as members of the DM System Science Team, they have responsibilities as described in \ref{sect:dm-sst-org}, which result in work that is more \textit{emergent} in nature.
To balance these two roles, the \gls{DMSS} is entitled to allocate up to 50\% of the Institutional Science Leads' time to Science Team work.
If any Science Team study should require a greater commitment, additional time must be negotiated and agreed with the institutional \glspl{T/CAM}s.
This arrangement is intended to ensure both a good working relationship between the T/CAMs and scientists, and that the \gls{DMSS} maintains sufficient support from the Science Team to deliver a system that meets the overall LSST science goals.

\subsection{DM Systems Engineering Team \label{sect:sysengt}}

The Systems Engineering Team is led by the DMPM (\secref{role:dmpm}) and looks after all aspects of systems engineering.
It is comprised of not only the Systems Engineer (\secref{role:sysengineer}), but also the Software Architect (\secref{role:softarc}), Operations Architect (\secref{role:opsarc}), \gls{DMSS} (\secref{role:dmps}), Pipeline Scientist (\secref{role:pipe}), Interface Scientist (\secref{role:dmis}), and the \gls{DDMPM} (\secref{role:ddmpm}).

While the product owners (\secref{role:prodo}) help DM to create products which are fit for purpose, the Systems Engineering Team must ensure we do it correctly. This group concerns itself with (sub)system wide decisions on architecture and software engineering.

The specific tasks of this group include:

\begin{itemize}
\item Formalize the product list for DM\footnote{In this sense, ``products'' are the software and systems which produce data products, rather than the data products themselves. See also \ref{sect:products}.}
\item Formalize the documentation tree for DM, defining which documents need to be produced for each product
\item Agree the process for tracing the baseline requirements verification and validation status.
\item Agree the formal versions of documents and software which form the technical baseline, individual items will go through the CCB for formal approval.  This includes upload to docushare.
\item Perform releases of software products --- including, but not limited to, the Science Pipelines --- as needed, using tooling provided by SQuaRE (\secref{sect:square}).
\item Debug unexpected build problems:
\begin{itemize}
  \item{Resolve issues related to the underlying build infrastructure directly;}
  \item{Pass off product-specific problems to the relevant product team.}
\end{itemize}
\item Maintain the build/packaging system e.g.  newinstall.sh, lsstsw, lsst\_build.
\end{itemize}

Some of these tasks are will be delegated to individual group members.
These individuals also are the conduit to/from the rest of the DM team to raise ideas/issues with the engineering approach.

\subsubsection{Communications}

The Systems Engineering Team will only physically meet to discuss specific topics: there will not be a regular meeting of the group outside of the one to one meetings with the DM project manager for the individuals in the group.
Discussions will be held via email until in person talks are required.

\subsection{DM Leadership Team \label{sect:dmlt}}

The purpose of the DM Leadership Team (DMLT) is to assist the DMPM  establish the scope of work and resource allocation across DM and ensure overall project management integrity across DM.
The following mandate established the DMLT:

\begin{itemize}
\item Charter/purpose
	\begin{itemize}
	\item Maintain scope of work and keep within resource allocation across DM
	\item Ensure overall project management integrity across DM
	\item Ensure Earned Value management requirements are met
	\end{itemize}
\item Membership
	\begin{itemize}
	\item Co-chaired by the \gls{DMPM} (\secref{role:dmpm}) and \gls{DMSS} (\secref{role:dmps})
	\item Lead Institution Technical/Control Account Managers (T/CAMs; \secref{role:tcam})
	\item Institutional Science or Engineering Leads (\secref{role:scilead})
	\item Members of the DM Systems Engineering Team (\secref{sect:sysengt})
	\end{itemize}
\item Responsibilities
	\begin{itemize}
	\item Prepares all budgets, schedules, plans
	\item Meets every week to track progress, address issues/risks, adjust work assignments and schedules, and disseminate/discuss general PM communications
	\end{itemize}
\end{itemize}

The DM Leadership Team and the DM Systems Engineering Team (\secref{sect:sysengt}) work in synchrony.
The DMLT makes sure the requirements and architecture/design are estimated and scheduled in accordance with LSST Project required budgets and schedules.

 \subsubsection{Communications}
A mailing list\footnote{\url{lsst-dmlt@listserv.lsstcorp.org}} exists for DMLT related messages.
On Mondays the DMLT hold a brief (30 to 45 minutes) telecon. This serves to:

\begin{itemize}
\item Allow the Project manager and DM Scientist  to pass on important project level information and general guidance.
\item Raise any blocking or priority issues across DM --- this may result in calling a splinter meeting to further discuss with relevant parties.
\item Inform all team members of any change requests (LCRs) in process at LSST level which may be of interest to or have an impact on DM
\item Check on outstanding actions on DMLT members
\end{itemize}

Face to Face meetings of DM are held twice a year; these are opportunities to:

\begin{itemize}
\item Discuss detailed planning for the next cycle
\item Discuss technical topics in a face to face environment
\item Work together on critical issues
\item Help make DM function as a team
\end{itemize}

\subsection{DM Change Control Board \label{sect:dmccb}}

The DMCCB has responsibility for issues similar to those of the LSST Change Control Board, but focused on the DM Subsystem.
The DMCCB reviews and approves changes to all baselines in the Subsystem, including proposed changes to the DM System Requirements (DMSR), reference design, sizing model, i.e. any LDM-series document.
The Technical Baseline, including software/hardware and documentation, is produced by DM and controlled by the DMCCB.
DMCCB validates that the form and content of the Technical Baseline is consistent with LSST project standards such as the Systems Engineering Management Plan (SEMP) \citeds{LSE-17}.

\begin{itemize}
\item Responsibilities:
        \begin{itemize}
        \item Determine when deliverables (controlled documents and software) are ready to be baselined (placed under configuration controlled status) or released. This include LDM series documents.
        \item Review and approve/reject proposed changes to baselined items - any LCR must go through the DMCCB before being submitted to the project CCB.
        \item Review all RFCs and approves \textit{flagged} RFCs prior to 'Adoption'
        \item Monitor and approve DM software releases
        \item Monitor the status of issues in the DM project on Jira
        \item Ensure that the DM Technical Baseline (LDM-xxx) follows LSST and DM configuration control processes.
        \end{itemize}
\item Membership:
        \begin{itemize}
        \item Core members:
                \begin{itemize}
                \item \gls{DMPM}
                \item \gls{DMSS}
                \item Systems Engineer, Chair (\secref{role:sysengineer}).
                \item Operations Architect
                \item Software Architect
                \item Release Manager, Secretary
                \end{itemize}
        \item Optional members (required when topics to discuss are relevant to their areas of expertise):
                \begin{itemize}
                \item \gls{DDMPM}, when \gls{DMPM} is not available
                \item Deputy \gls{DMSS}, when \gls{DMSS} is not available
                \item Pipeline Scientist
                \item Science Pipelines / Interface Scientist
                \item \glspl{T/CAM}, who can delegate to their deputies
                \end{itemize}
	\item For on-line virtual meetings, if a consensus or quorum is not reached within one week, the \gls{DMPM} will make a unilateral decision
        \item \gls{DMPM} can also make unilateral decisions in cases of urgency. In that case DMCCB will assess the change \textit{a posteriori}.
	\end{itemize}
\end{itemize}

The DMCCB will meet, physically or virtually, every week for 30 minutes. Agenda will be available beforehand.
Urgent decisions can be taken offline, outside the weekly meeting, in a modality to be defined by the DMCCB itself (email or slack channel).

All RFCs that implies one of the following changes:

\begin{itemize}
\item Changes to controlled documents
\item API changes to the codebase, including deprecation
\item Data model changes
\end{itemize}

need to be \textit{flagged} and therefore approved by the DMCCB, as detailed in the \href{https://developer.lsst.io/communications/rfc.html#rfc-exceptions}{Developer Guide}.


\subsection{DM Science Validation Team}
\label{sect:dmsvt}

The DM Science Validation Team guides the definition of, and receives the products of, science validation and dress rehearsal activities, following the long-term roadmap described in \citeds{LDM-503}.
Decisions on the strategic goals of these activities are made in conjunction with the \gls{DMSS} and \gls{DMPM}.

The DM Science Validation Team is chaired by the DM Science Validation Scientist (\secref{role:dmsvs}).
Its membership includes the DM Pipelines Scientist (\secref{role:pipe}) and the various Institutional Science/Engineering Leads (\secref{role:scilead}).
Depending on the activities currently being executed, other members of the System Science Team (\secref{sect:dmsst}), the wider DM Construction Project, and/or external experts may be temporarily added to the team.


\subsection{Middleware Team \label{sec:middleware}}

The Middleware Team is responsible for delivering the Data Butler and pipe\_base task framework, including supporting infrastructure to make it possible to deploy them at-scale in the Data Facility in support of Alert and Data Release Production pipeline execution.

The Middleware Team has a Product Owner (Robert Gruendl, NCSA at time of writing) and Manager (\secref{role:mwlead}).
However, it does not have a permanent staff; rather it draws on effort from across the Alert Production (\secref{sect:ap}), Data Release Production (\secref{sect:drp}), Data Access Services (\secref{sect:dax}), and Data Facility (\secref{sect:ldf}) groups, as well as other members of the subsystem as necessary.
Effort allocation is agreed between the Middleware Manager and the T/CAMs of the various institutes.

\subsection{Science Platform Team \label{sec:sciplat}}

The Science Platform Team is responsible for delivering the three aspects of the LSST Science Platform, as described in \citeds{LDM-542}.

The Product Owner for the Science Platform is the \gls{DMSS}, supported by the Science Platform Scientist (\secref{role:scip}).
The team is managed by the Science Platform Manager (\secref{role:lsplead}).
They coordinate effort across the subsystem, drawing primarily on the Data Access Services (\secref{sect:dax}), Data Facility (\secref{sect:ldf}) and SQuaRE (\secref{sect:square}) teams.
